\documentclass[10pt,twocolumn]{IEEEtran}
\usepackage{amssymb,amsmath,color,graphicx, amsbsy}
\usepackage{graphicx}
\usepackage{multirow}
\usepackage{subfigure}
\usepackage{cite}
\usepackage{algorithm}
\usepackage{algorithmic}
\usepackage{amsmath}
\usepackage{amsthm}
\usepackage{amsfonts}
\usepackage{amssymb}

\usepackage{graphicx}
\usepackage{epstopdf}
\usepackage{makeidx}
\usepackage[noprefix]{nomencl}
\usepackage{textcomp}
\makeindex
\makenomenclature
\makeglossary
\setcounter{MaxMatrixCols}{10}
\hyphenation{op-tical net-works partially-conduc-tor IEEEtran}
\DeclareMathOperator*{\argmax}{arg\,max}
\DeclareMathOperator*{\argmin}{arg\,min}
\newtheorem{theorem}{Theorem}
\newtheorem{assumption}{Assumption}
\newtheorem{acknowledgement}[theorem]{Acknowledgement}
\newtheorem{axiom}[theorem]{Axiom}
\newtheorem{case}[theorem]{Case}
\newtheorem{claim}[theorem]{Claim}
\newtheorem{conclusion}[theorem]{Conclusion}
\newtheorem{condition}[theorem]{Condition}
\newtheorem{conjecture}[theorem]{Conjecture}
\newtheorem{corollary}{Corollary}
\newtheorem{criterion}[theorem]{Criterion}
\newtheorem{definition}{Definition}
\newtheorem{example}[theorem]{Example}
\newtheorem{exercise}[theorem]{Exercise}
\newtheorem{lemma}{Lemma}
\newtheorem{notation}[theorem]{Notation}
\newtheorem{problem}[theorem]{Problem}
\newtheorem{proposition}{Proposition}
\newtheorem{remark}[theorem]{Remark}
\newtheorem{solution}[theorem]{Solution}
\newtheorem{summary}[theorem]{Summary}
\newtheorem{game}{Game}

\begin{document}

\title{\fontsize{22.5pt}{27pt}\selectfont{Detailed Modeling of Battery in Optimal Energy Storage Operation:  Cell Removal Impact on Capacity }\vspace{0.05cm}}

\author{Zach Taylor, \IEEEmembership{Student Member,~IEEE}, Hossein Akhavan-Hejazi, \IEEEmembership{Member,~IEEE}, and Hamed Mohsenian-Rad, \IEEEmembership{Senior Member,~IEEE}

\thanks{Z. Taylor, H. Hejazi, and H. Mohsenian-Rad are with the Department of Electrical Engineering, University of California, Riverside, CA, USA.  The corresponding author is H. Mohsenian-Rad, e-mail: hamed@ece.ucr.edu.} \vspace{-0.5cm}}

\maketitle

\vspace{-0.4cm}

\begin{abstract}
Abstract Here
\vspace{0.3cm}
\textbf{\emph{Keywords}}: keyword1, keyword2, keyword3.
\end{abstract}

\vspace{-0.1cm}

\section*{Nomenclature}
\addcontentsline{toc}{section}{Nomenclature}

%In this paper we denote all  sets by caligraphic capital letters, all continuous decision variables with capital letters, and all constants with lower-case letters.
%We use subscripts to differentiate variables or constants that have similar qualities, e.g. voltage or power.


\begin{IEEEdescription}[\IEEEusemathlabelsep\IEEEsetlabelwidth{$\Phi, \Psi, \Omega, \Gamma, L$}]

%For each set, please add the notation itself to the list and introduce it directly. I could identify \mathcal{N}, \mathcal{T}, and \mathcal{Q}. Each one will have one line. There might be more sets. Please double check.

\item[$n, m$] Index variables for cell number
\item[$t,\tau$] Index variables for timeslot
%\item[$o, q,  u, w$] Index variables for discretization
\item[$ \mathcal{N}$,$\mathcal{T}$] Sets of all battery cells, and timeslots
%\item[$\mathcal{T}$] Set of all timeslots
%\item[$\mathcal{Q}$] Set of segments in binary expansion
\item[$\overline{n}$] Number of battery cells in series
%\item[$\delta, \sigma, \zeta$]  Length of binary expansion segments
\item[$c_0$]  Initial charge of a battery cell
\item[$r$]  Internal resistance of a battery cell
\item[$P_o$]  Output power of a battery cell
%\item[$P_{in}$]  Internal power generated/absorbed by a cell
\item[$V_{oc}$]  Open circuit voltage of a battery cell
\item[$V_{o}$]  Terminal voltage of a battery cell
\item[$I$]  Current into a battery cell
\item[$C$]  Current available capacity of a battery cell
\item[$\overline{v}, \overline{i}, \overline{c}$] Maximum voltage, current, capacity limits
\item[$\underline{v}, \underline{i}, \underline{c}$] Minimum voltage, current, capacity limits
\item[$f(\cdot)$] Function to relate $V_{oc}$ and $C$. 
%\item[$a$] Slope of linear function of $f(\cdot)$
%\item[$l$] Look up table for $f(\cdot)$
%\item[$B, D, F$] Binary variables
%\item[$\Phi, \Psi, \Omega, \Gamma, L$] Auxiliary variables
%\item[$\alpha, \beta, \gamma$] Discretization parameters
\vspace{0.25cm}
\end{IEEEdescription}



\section{Introduction}

\color{red}
Background and Motivation

Literature Review

Contributions

\color{black}

We  argue that, if the reduced capacity of a battery pack is due to a single or few cells with lower capacity, then it could be that removing the degraded cell(s) actually increases the battery pack capacity. We then present a method in order to identify the cells, if any, that need to be removed to enhance the usable capacity of the battery pack.

\section{Battery Pack Capacity Formulation}

A typical battery pack is composed of many cells that are connected in series and strings of cells that are connected in parallel. 
Despite all efforts the cells are not exactly the same and particularly over time, not all cells within a battery pack will have the same conditions.
Thus, the voltage trajectory during charge/discharge will vary for different cells in a pack.
For instance, consider the voltage profiles of a lab-scale BESS,  shown in Fig. \ref{fig:packoperation}, during one cycle of discharge and charge.
While the 12 cells in this single battery module are discharged and charged with the same current, the voltage of some cells decline and incline faster than others.
%Additionally, when the pack begins to charge, some of the cells may already have some energy stored.
Thus, the available capacity of a battery pack is often affected by the capacity variations  across different cells.


When the battery cells in series  are discharged with the same current, the voltage on the terminals of those cells that are in poorer state of health conditions  declines faster.
Thus they reach to the minimum allowed voltage sooner than the other series cells in the  string.
At this point, often the battery management systems stop the discharge operation for safety of the pack.
Accordingly, the capacity limit of the cells with lowest capacity limits the battery operation, hence, some of the battery pack capacity remains unused.

\begin{figure}[t]
\centering
\vspace{-0.2cm}
{\scalebox{.5}{\includegraphics*{./battery_cycling2.eps}}} \vspace{-0.1cm}
\caption{The voltage profiles of the cells in a battery pack.}
\vspace{-0.3cm}
\label{fig:packoperation}
\end{figure}

\subsection{Capacity Modeling of Cells  in Series.}
Consider a battery pack with $\overline{n}$  number of cells in series. 
Let us denote the charge level of cell $n$ at time slot $t$  during the battery operation with $C[t,n]$.
We also denote the capacity limit of cell $n$ with $\overline{c}_n$.
Naturally, when the battery pack is discharged/charged  the power drawn/stored in the battery module at any time slot $t \in \mathcal{T}$, denoted by $P_{out}[t]$,  is the sum of the power drawn/stored in each individual cell $n$, denoted by $P_{o}[t,n]$:
\begin{equation}
P_{out}[t]=\sum_{n \in \mathcal{N}}  P_{o}[t,n].%,  \  \  \  \forall t \in \mathcal{T}.
\label{Pmodule}
\end{equation}

\noindent Here $P_{o}[t,n]$ can take both positive and negative values, where negative values indicate discharging of the battery.
If we assume a simple cell model of a voltage source and internal resistance $r$, for each cell the stored energy in the cell is equal to:

\begin{equation}
C[t,n]=C[t-1,n] +  \tau(P_{o}[t,n] - r[n]I[t])
\label{energycell}
\end{equation}

The cell charge level is directly related to the open circuit voltage $V_{oc}[t,n]$ of the cell by an increasing function $f(\cdot)$.
The open circuit voltage is the voltage of the cell when there is no current applied, and all of the transients in voltage have settled.
Thus $C[t,n]$ is at its maximum $C[t,n]=\overline{c}$  when $V_{oc}[t,n]=\overline{v}$.
In practice, $V_{oc}[t,n]$ is difficult to measure and the terminal voltage $V_o[t,n]$ is often the way the battery operation is limited.
For example, a cell cannot be charged beyond its voltage limit $\overline{v}$ or discharged beyond its voltage limit $\underline{v}$.
If any cell reaches these voltage limits, the battery operation must cease.
Thus, since we cannot overcharge a single cell, every cell is limited to the capacity of the smallest capacity cell; even if the cells are balanced.
Due to this we can find the maximum usable energy storage capability for a pack of cells to be:

\begin{equation}
\overline{n} \ \underset{n\in \mathcal{A}}{min} \ \overline{c}[n]
\label{eq:lhscap}
\end{equation}

Where $\mathcal{A}$ is the set of all cells in the pack and $\overline{n}$ is the number of cells, i.e, the cardinality of set $\mathcal{A}$. 
Essentially, a battery pack's usable capacity is the capacity of the cell with minimum capacity, times the number of cells.




\subsection{Diagnosis for Cell Removal }
Normally  a cell is not removed because of being out of balance,  rather the battery  pack would require (SoC) balancing.
Accordingly,  when we perform the  analysis for necessity of cell removal, we assume the pack is balanced and the minimum SoC of each cell is zero Wh.



Let us define a new set, $\mathcal{B}_x$ to be the set of all cells except the x lowest capacity cells.
%Typically x is one; but it could also be greater than one if there are multiple cells with the lowest capacity.
If we build a pack out of the cells in set $\mathcal{B}_x$, the new pack would have $\overline{n} - x$ cells, thus the new capacity is:

\begin{equation}
(\overline{n}-x) \underset{m\in \mathcal{B}_x}{min} \ \overline c[m]
\label{eq:rhscap}
\end{equation}

The number of cells was lowered by $x$, but now the minimum capacity cell(s) have been removed; therefore the min term in (\ref{eq:rhscap}) is now greater than the min term in (\ref{eq:lhscap}). The question becomes, will this new capacity gain be greater than the capacity loss from the removed cell?

Mathematically, this question can be formulated as whether the following inequality holds:

\begin{equation}
\overline{n} \  \underset{n\in \mathcal{A}}{min} \ \overline c[n] < (\overline{n}-x) \ \underset{m\in \mathcal{B}_x}{min} \ \overline c[m] \ \  \forall x
\label{eq:removecellineq}
\end{equation}

If the inequality in (\ref{eq:removecellineq}) holds, then removing the $x$ lowest capacity cells would actually improve total usable energy in the pack. 
%We can visualize this using our test data. As part of the calibration/training step, each cell that reaches zero energy stored will reestimate its parameters. These parameters include $\overline c$ and $r[n]$ (internal resistance). Since we have an estimate for $\overline c$ , we can use it in the above equations to evaluate how helpful a cell is to the total pack. We know that no cell can use more capacity than the lowest cells capacity.



%%%%%%%%%%%%%%%%%%%%%%
\section{Experimental Results, Model Validations,\\ and Operation Performance Assessment} \label{sec:results}
%%%%%%%%%%%%%%%%%%%%%%%%%%%%%%%%


\subsection{Initial Capacity Testing} \label{sec:capacity_test_1}

The Power Hardware in Loop (PHIL) tesbed presented in \cite{NAPS_RTDS} was used to charge and discharge the battery using constant current. The battery in this test consists of 40Ah GBS Lithium Iron Phosphate cells \cite{elitepower}. However, these cells are in the used condition, and several of them have degaded over time. In addition, the pack is also unbalanced, meaning each cell is at a slightly different charge level. During the test the amount of energy at the terminals of the battery pack was recorded while the pack was charged from empty to full. This energy value is the current available energy storage capacity of the pack. 

\subsection{Cell Capacity Estimation}

In order to calculate the values of $\overline{c}$ for each cell, a cell model that estimates maximum capacity must be fit to the operational data gathered in \ref{sec:capacity_test_1}.
It is not a simple matter to calculate $\overline c$.
\color{red}
Discussion on how to estimate maximum capacity goes here. Which model should we use? We can use the linear and LUT methods from our journal paper, but we will have to formulate them again or reference another method/paper that also estimates cell capacity.
\color{black}

After running the parameter estimation we have the following estimated capacities for each cell shown in Fig. \ref{fig:cell_removal1}.
  
\begin{figure}[t]
\centering
\vspace{-0.2cm}
{\scalebox{.5}{\includegraphics*{cell_removal_figure.eps}}} \vspace{-0.1cm}
\caption{Estimated capacity for each cell in the pack (a) unsorted and (b) sorted from least capacity to most.}
\vspace{-0.3cm}
\label{fig:cell_removal1}
\end{figure}

\subsection{Cell Removal}

The usable capacity with no change is:
$\overline{n} min(C_{max}[n]) = 12 \times 25 = 300WHr$
By using the formula above for every value of $x$ we can find out the usable capacity for each scenario, as a function of $x$, from removing only the lowest cell to removing every cell (which of course is obviously zero). This chart is shown below. We would then simple choose the greatest capacity scenario.

  
\begin{figure}[t]
\centering
\vspace{-0.2cm}
{\scalebox{.5}{\includegraphics*{cell_removal_figure2.eps}}} \vspace{-0.1cm}
\caption{Usable capacity vs x}
\vspace{-0.3cm}
\label{fig:cell_removal2}
\end{figure} 

This chart tells us that removing the bottom 3 cells would result in the maximum usable capacity for the entire pack. Almost double the current usable capacity is available if we simply remove the underperforming cells.

\subsection{Final Capacity Testing}
After the above cells had been removed the entire pack was balanced and the constant current cycle test from Section \ref{sec:capacity_test_1} was repeated.
\color{red}
Same test as in \ref{sec:capacity_test_1}, used to show the improved performance, should be more than capacity found in \ref{sec:capacity_test_1}
\color{black}
\section{Conclusions}

\color{red}CONCLUSION HERE\color{black}

\bibliographystyle{IEEEtran}
\bibliography{IEEEabrv,refs}
\end{document}
